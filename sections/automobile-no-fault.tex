\section{Automobile No-Fault Insurance}

\subsection{Hager, ``No-Fault Drives Again: A Contemporary Primer''}

\begin{enumerate}
    \item 26 states have some sort of no-fault system.
    \item Arguments in favor: rational and efficient compensation for victims.
    \item Arguments against: weakens deterrence and corrective justice.
    \item ``Pure'' no-fault bars all auto lawsuits.
    \item ``Partial'' no-fault allows tort actions only above a certain 
    threshold, either verbal (in which eligible injuries are defined) or 
    monetary (victims can bring tort claims for damages above a certain 
    level).
    \item ``Choice'' no-fault allows drivers to choose either no-fault coverage 
    or tort lawsuit rights.
    \item Congressional ``neo-partial'' no-fault allows victims to sue if the 
    driver was intoxicated or engaged in intentional misconduct.
    \item ``Cost and compensation advantages are greatest when no-fault is 
    closest to pure.''\footnote{Casebook p. 693.}
    \item ``Though studies are mixed, it seems unlikely that no-fault 
    seriously undermines road safety. It is also unlikely to augment 
    it.''\footnote{Casebook p. 694.}
\end{enumerate}
