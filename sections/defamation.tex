\section{Defamation}

\subsection{Defamatory Assertion of Fact}

\begin{enumerate}
    \item Common law defamation consists of defamatory assertions of fact 
    against the plaintiff negligently or intentionally published. Under the 
    common law, defendants who published defamatory material were held 
    strictly liable. Now, there are constitutional requirements that the 
    defendant act negligently or recklessly in some cases.\footnote{Casebook p. 697.}
    \item The Restatement (Second) defines a defamatory statement as one that 
    ``tends so to harm the reputation of another as to lower him in the 
    estimation of the community or to deter third persons from associating or 
    dealing with him.''\footnote{Casebook p. 697.}
    \item ``Only assertions of fact that can be proven false are subject to 
    liability for defamation.''\footnote{Casebook p. 702.; \emph{Milkovich v. 
    Lorain Journal Co.}, 497 U.S. 1 (1990).}
    \item \emph{Libel per se}: a statement is defamatory on its face.
    \item \emph{Libel per quod}: a statement is defamatory in the context of 
    other information (e.g., ``A married B'' is not defamatory unless the 
    listener knows A is already married to C).\footnote{Casebook p. 705.}
    \item Defamation must be measured by the standard of the ``right-thinking 
    person.''\footnote{Casebook pp. 705--06.}
    \item \emph{Colloquium}: if it is not obvious that the statement refers to 
    the plaintiff, the plaintiff must prove colloquium (i.e., necessary 
    facts).\footnote{Casebook p. 706.}
\end{enumerate}

\subsubsection{\emph{Kaplan v. Newsweek Magazine, Inc.}}

\begin{enumerate}
    \item Kaplan sued Newsweek for libel for publishing an article criticizing his class 
    at Stanford. The district court dismissed the claim.
    \item The Ninth Circuit affirmed, holding that the statements were 
    non-defamatory or statements of opinion rather than of fact.
\end{enumerate}

\subsubsection{\emph{Kaelin v. Globe Communications, Inc.}}

\begin{enumerate}
    \item Kaelin sued the National Examiner over a headline that implied that 
    the police thought he had murdered O.J. Simpson's wife and her friend.
    \item The court held that a jury question existed. The measure is whether 
    an ``average reader'' would be likely to interpret the statement as the 
    plaintiff claimed. ``So long as the publication is reasonably susceptible 
    of a defamatory meaning, a factual question for the jury 
    exists.''\footnote{Casebook p. 701.}
\end{enumerate} 

\subsubsection{Defamation in Fiction: \emph{Bindrim v. Mitchell}}

Works of fiction can be defamatory if a reasonable person would identify a 
character as a real person.

\begin{enumerate}
    \item Mitchell attended Bindrim's ``Nude Marathon'' therapy sessions on 
    the condition that she would not write about them. She later published a 
    book containing similar events. Bindrim sued.  \item The jury found for 
    Bindrim.
    \item Mitchell argued that the character in the novel could not be 
    identified as Bindrim.
    \item Bindrim argued further that works of fiction cannot be defamatory.  
    The court rejected this argument, holding that ``[t]he test is whether a 
    reasonable person, reading the book, would understand that the fictional 
    character therein pictured was, in actual fact, the plaintiff acting as 
    described.''\footnote{Casebook p. 704.}
    \item Whether the novel was defamatory was a jury question. The jury's 
    verdict cannot be overturned. Affirmed.
\end{enumerate}

\subsection{Libel Versus Slander}

\begin{enumerate}
    \item \textbf{Slander}: spoken. Except for four categories of slander per 
    se,\footnote{See Restatement (Second) \S\ 570 below.} slander 
    requires proof of special (i.e., pecuniary) damages as a prerequisite 
    to recover for nonpecuniary damages (e.g., pain and suffering).
    \item \textbf{Libel}: recorded. There is no requirement to prove special 
    damages to recover for emotional distress.
\end{enumerate}

\subsubsection{Restatement (Second)}

\paragraph{\S\ 568: Libel and Slander Distinguished}

\begin{enumerate}
    \item ``...no respectable authority has ever attempted to justify the 
    distinction [between libel and slander regarding the special damages 
    requirement] in principle...''\footnote{Casebook p. 707.} The distinction 
    is rooted in the divide between ecclesiastical and common law courts, and 
    ``although indefensible in principle, was too well established to be 
    repudiated.''\footnote{Casebook p. 708.}
\end{enumerate}

\paragraph{\S\ 568A: Radio and Television}

\begin{enumerate}
    \item Defamatory material broadcasted or disseminated is libel.
\end{enumerate}

\paragraph{\S\ 569: Liability without Proof of Special Harm---Libel}

\begin{enumerate}
    \item There is no need to prove special harm resulting from publication.
\end{enumerate}

\paragraph{\S\ 570: Liability without Proof of Special Harm---Slander}

\begin{enumerate}
    \item Slander \emph{does} require proof of special harm unless the 
    material alleges:\footnote{Casebook p. 710.}
    \begin{enumerate}
        \item A criminal offense.
        \item A loathsome disease.
        \item Matter incompatible with his business, trade, or profession.
        \item Serious sexual misconduct.
    \end{enumerate}
\end{enumerate}

\paragraph{\S\ 575: Slander Creating Liability Because of Special Harm}

\begin{enumerate}
    \item Slander creates liability on the basis of special harm.
    \item Special harm is the loss of something with economic or pecuniary 
    value.
    \item The special harm requirement has been expanded to include loss of 
    companionship, etc., when the companionship has a monetary value. 
    \item Loss of the material advantages of hospitality can constitute 
    special harm.
\end{enumerate}

\paragraph{\S\ 623: Emotional Distress and Resulting Bodily Harm}

\begin{enumerate}
    \item One who is liable for libel or slander can also be liable for 
    causing emotional distress.
\end{enumerate}

\subsection{Publication}

\subsubsection{Publication Requirement for Libel: \emph{Weidman v. Ketcham}}

\begin{enumerate}
    \item A post office employee sent a postcard in an envelope to a man 
    demanding payment for stolen apples. The man sued for libel.
    \item At trial, the jury found for the plaintiff, but the court granted 
    the defendant's motion to set aside the verdict. The appellate court 
    reversed.
    \item The defendant argued that there had been no publication and 
    therefore there could be no liability for libel. The court held that 
    without evidence of publication (and in this case, without evidence that 
    the postmaster knew the identity of the recipient) there could be no 
    liability for libel. Reversed.
\end{enumerate}

\subsection{Constitutional Culpability Requirement}

\begin{enumerate}
    \item Public officials must show ``\emph{New York Times} malice.''
    \item Public figures must also meet the public official standard.
    \item Private plaintiffs in public controversies do not need to prove 
    \emph{New York Times} malice, though they must prove fault (``generally 
    understood to mean negligence toward the truth'').\footnote{Casebook p. 
    727 n. 2.}
    \item The plaintiff now has the burden of proving that the defamation was 
    false. (Under the common law, the defendant had the burden of proving 
    truth.)\footnote{Casebook p. 731.}
\end{enumerate}

\subsubsection{Defamation of Public Officals: \emph{New York Times Co. v. Sullivan}}

There is a higher burden of proof for defamation of public officials. In 
addition to falsehood, the plaintiff must prove that the defendant knew the 
statements were false or acted recklessly towards the truth.

\begin{enumerate}
    \item The \emph{New York Times} ran a full page ad describing civil rights 
    abuses of several public officials in Montgomery, Alabama. It contained 
    several factual errors which were apparently unwitting.
    \item Sullivan, a Commissioner of Mongomery, Alabama, sued for libel. The 
    state trial court found for the plaintiff and the Alabama Supreme Court 
    affirmed.
    \item Justice Brennan:
    \begin{enumerate}
        \item Learned Hand: The First Amendment ``presupposes that right conclusions are 
        more likely to be gathered out of a multitude of tongues, than through 
        any kind of authoritative selection. To many this is, and always will 
        be, folly; but we have staked upon it our all.''\footnote{Casebook p. 
        716.}
        \item A rule requiring absolute factual accuracy in criticism of 
        public officials leads to self-censorship.
        \item Proving defamation of a public official requires ``actual 
        malice'' in addition to falsehood.
        \item Reversed.
    \end{enumerate}
    \item ``\emph{New York Times} malice'' requires that the defendant knew 
    the statements were false or acted recklessly towards the truth. It is 
    distinct from common law malice, which refers to hatred, ill will, or 
    reckless disregard.\footnote{Casebook p. 719.}
\end{enumerate}

\subsubsection{Defamation of Public Figures: \emph{Gertz v. Robert Welch, Inc.}}

Public figures, like public officials, must also prove \emph{New York Times} 
malice to recover damages for defamation. In matters of public concern, 
private individuals cannot recover against publishers without proving actual 
malice.

\begin{enumerate}
    \item A Chicago policeman, Nuccio, shot a youth, Nelson. Nelson's family 
    retained an attorney, Getz, to represent them in civil litigation against 
    Nuccio. 
    \item Robert Welch, Inc. was the publisher of the \emph{American Opinion}, a 
    magazine of the John Birch society. It published an article rife with 
    falsehoods accusing Gertz of participating in a communist conspiracy 
    against the police.
    \item Justice Powell:
    \begin{enumerate}
        \item The issue is whether publishers have a constitutional privilege 
        against liability for defamation against people who are not public 
        figures or public officials.
        \item States can define their own standards for protections of 
        publishers from liability, as long as the standard requires knowledge 
        of falsehood or reckless disregard for the truth.
        \item Welch also argued that Gertz was a ``public figure'' and 
        therefore that the \emph{New York Times} standard should apply. The 
        Court here held that protections could extend to public figures who 
        ``voluntarily inject'' themselves into public controversies. In this 
        case, however, Gertz was not a public figure, so he was entitled to 
        the stricter protections afforded private citizens.
        \item Reversed.
    \end{enumerate}
\end{enumerate}

\subsubsection{Defamation in Private Affairs: \emph{Dun \& Bradstreet, Inc. v. 
Greenmoss Builders, Inc.}}

\begin{enumerate}
    \item Dun \& Bradstreet, a credit reporting agency, mistakenly 
    misrepresented Greenmoss's assets and liabilities.
    \item Greenmoss sought compensatory and punitive damages in Vermont state 
    court. The jury 
    returned \$50,000 in compensatory damages and \$300,000 in punitive 
    damages.
    \item On appeal, Dun \& Bradstreet argued that plaintiffs should not be 
    able to recover damages in defamation actions without showing actual 
    malice.
    \item Justice Powell: ``permitting recovery of presumed and punitive 
    damages in defamation cases absent a showing of `actual malice' does not 
    violate the First Amendment when the defamatory statements do not involve 
    matters of public concern.''
    \item Affirmed.
\end{enumerate}

\subsection{Privileges}

\begin{enumerate}
    \item At common law, \textbf{absolute privilege} extends to statements in 
    legislatures, in judicial proceedings, and by high executive officials in 
    official capacities. Private conversations between spouses are also 
    absolutely protected from defamation liability.
    \item A \textbf{qualified privilege} exists where a speaker tries to 
    protect the interest of the person he is speaking to---e.g., a recommender 
    of a job applicant to the employer. The privilege is qualified because it 
    can be negated through bad faith, recklessness, or excessive 
    communication.
    \item The \textbf{fair and accurate report} privilege allows republication 
    of defamatory material in the context of governmental proceedings and 
    public meetings. Whether the privilege extends to defamation beyond 
    official proceedings is in controversy.
    \item Retraction can mitigate defamation liability.
    \item 47 U.S.C. \S\ 230 grants immunity to Internet providers for 
    republishing defamatory material from third-party providers.
\end{enumerate}

\subsubsection{Speech or Debate Clause: \emph{Hutchinson v. Proxmire}}

The Speech or Debate Clause protects members of Congress from defamation 
actions over statements they make in leglslatures. The privilege does not 
extend to statements in communications to constituents.

\begin{enumerate}
    \item Senator Proxmire instituted the ``Golden Fleece'' award, criticizing 
    examples of what he considered wasteful government spending. One of the 
    recipients of the award, Hutchinson, was a behavioral research scientist 
    who studied anger in animals. Proxmire announced the award in a speech 
    published in the congressional record and in mailings to his constituents. 
    Hutchinson sued Proxmire and his legislative assistant for libel.
    \item The Speech or Debate Clause provides that members of Congress 
    ``...shall in all Cases, except Treason, Felony, and Breach of the Peace, 
    be privileged from Arrest during their attendance at the Session of their 
    Respective Houses, and in going to and from the same; and for any Speech 
    or Debate in either House, they shall not be questioned in any other 
    Place.''\footnote{Art. I, \S\ 6, cl. 1.}
    \item Proxmire argued that the Speech or Debate Clause protects both his 
    speech and his newsletters. He argued further that the newsletters were 
    protected under the ``informing function'' of Congress.
    \item Justice Burger: while the clause protects congressional speeches, 
    ``[w]e are unable to discern any `conscious choice' to grant immunity for 
    defamatory statements scattered far and wide by mail, press, and the 
    electronic media.''\footnote{Casebook p. 735.} Also, Proxmire misconstrued 
    the meaning of ``inform.'' The informing function refers to Congress's 
    ability to inform itself, not for it to inform the public.
\end{enumerate}

\subsubsection{News Media Privilege: \emph{Brown v. Kelly Broadcasting Co.}}

News media do not have a ``public interest'' privilege.

\begin{enumerate}
    \item Brown sued Kelly Broadcasting for defamation in a TV news report 
    suggesting that Brown, a contractor, did shoddy work for recipients of 
    government home improvement loans.
    \item Cal. Civ. Code \S\ 47(3) privileges communications made without 
    malice when the speaker and recipient share a common interest. Kelly 
    argued that \S\ 47(3) created a ``public interest'' protection for news 
    media. The court rejected this argument because it would grant protection 
    to almost all news media communications and there was no evidence of 
    legislative intent to create such a broad scope.
    \item Mansfield: ``Whenever a man publishes he publishes at his peril.'' 
    Holmes: ``If the publication was libellous the defendant took the 
    risk.''\footnote{Casebook p. 740.}
\end{enumerate}
