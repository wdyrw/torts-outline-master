\subsection{Negligent Misrepresentation and Economic Loss}

\begin{enumerate}
    \item Courts are generally hesitant to award damages for pure economic 
    loss. Damages must usually arise from personal injury or property damage. 
    ``In essence, there is no legal duty under negligence to refrain from 
    causing pure economic loss.''\footnote{Casebook p. 405.}
    \item Under the Restatement (Second) definition, \textbf{negligent 
    misrepresentation} occurs when a defendant supplies false information if 
    the plaintiff justifiably relied on it.\footnote{Casebook p. 406.}
    \item Most courts require a business relationship between the parties for 
    recovery for negligent misrepresentation.
    \item The Second (Restatement) view allows third party recovery if the 
    information provider intended to supply them with the information or knows 
    the recipient intends to use it for a similar transaction.
    \item \textbf{Most courts do not allow recovery of pure economic loss 
    beyond misrepresentation in business or professional relationships.} 
    \emph{J'Aire}, below, is unusual.
\end{enumerate}

\subsubsection{Negligent Misrepresentation: \emph{Bily v. Arthur Young \& 
Co.}}

Third parties can recover for negligent misrepresentation if they were among 
the information provider's intended beneficiaries.

\begin{enumerate}
    \item Plaintiffs invested in the Osborne Computer Company. Osborne hired 
    Arthur Young to handle its accounting. Plaintiffs alleged negligent 
    misrepresntation in Arthur Young's audit statement, claiming that the 
    audit stated a slim operating profit when in fact the operating loss was 
    more than \$3 million, and that Arthur Young discovered material 
    weaknesses in the company's accounting controls but failed to report them. 
    The plaintiffs won on a general negligence rule.
    \item The California Supreme Court here weighed three approaches to 
    negligent misrepresentation:
    \begin{enumerate}
        \item \textbf{Privity}: Cardozo in \emph{Ultramares} held that 
        auditors are only liable to third parties with a relationship to the 
        auditor ``akin to privity.''\footnote{Casebook p. 399.}
        \item \textbf{Foreseeability}: Third parties can recover from the 
        auditor if their reliance on the audit was foreseeable.
        \item \textbf{Intended beneficiaries}: The Restatement (Second) holds 
        that third parties can recover from the auditor if they justifiably 
        rely on the audit.
    \end{enumerate}
    \item The court adopted the intended beneficiaries rule. Reversed.
\end{enumerate}

\subsubsection{Economic Loss: \emph{J'Aire Corp. v. Gregory}}

Contractors are liable to tenants if their work causes foreseeable injury to a 
tenant's business.

\begin{enumerate}
    \item J'Aire operated a restaurant at the Sonoma County Airport. Gregory 
    was contracted to do construction work, which it failed to complete in a 
    timely fashion. J'Aire claimed the delays caused it to lose customers. 
    Gregory won a demurrer.
    \item The California Supreme Court held ``that a contractor owes a duty of 
    care to the tenant of a building undergoing construction work to prosecute 
    that work in a manner which does not cause undue injury to the tenant's 
    business, where such injury is reasonably foreseeable.''\footnote{Casebook 
    p. 411.} The court strongly emphasized the foreseeability requirement. 
    Reversed.
\end{enumerate}
