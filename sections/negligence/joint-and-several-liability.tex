\subsection{Joint and Several Liability}

\begin{enumerate}
    \item Defendants who share responsibility for a tort are \textbf{jointly 
    and severally liable} for damages, which means each is fully responsible 
    for the entire injury.
    \begin{enumerate}
        \item \textbf{Joint liability}: multiple parties share liability.
        \item \textbf{Several liability}: liability that is separate and 
        distinct from another's liability.\footnote{Black's Law.}
    \end{enumerate}
    \item If one person aids or encourages another to cause the injury, the 
    two are \textbf{acting in concert}. The assistance must be tortious.
    \item If multiple actors committing independent torts cause a single 
    individual injury, all tortfeasors will be held jointly and severally 
    liable.
    \item When more than one tortfeasor was available to pay, courts would 
    traditionally divide the payment by the number of defendants (e.g., two 
    defendants would each pay 50\%). Courts often now use \textbf{comparative 
    indemnification} to allocate damages in proportion to fault.
    \item Comparative indemnification is compatible with joint and several 
    liability because if one of the defendants can't pay, the others must fill 
    the gap proportionally.
    \item \textbf{Contribution} allows a defendant who paid all of the damages 
    to recover from additional defendants who are at fault. By contrast, 
    \emph{indemnification} involves a total shift in liability to a more 
    culpable party.\footnote{Casebook p. 484.}
\end{enumerate}

\subsubsection{Comparative Indemnity and Cross-Claims: \emph{American 
Motorcycle Association v.  Superior Court}}

The California Supreme Court establishes comparative indemnity and the ability 
for defendants to make cross-claims against any person from whom they seek 
indemnity.

\begin{enumerate}
    \item Gregos was injured in a motocross race. He sued the American 
    Motorcycle Association and the Viking Motorcycle Club. AMA cross-claimed 
    against Gregos's parents, alleging the negligently allowed him to 
    participate and seeking indemnification if AMA was found liable.
    \item The Supreme Court of California held:
    \begin{enumerate}
        \item Comparative negligence does not abolish joint and several 
        liability. ``[E]ach tortfeasor whose negligence is a proximate cause 
        of an indivisible injury remains individually liable for all 
        compensable damages attributable to that injury.''\footnote{Casebook 
        p. 470.}
        \item The equitable indemnity doctrine (which divided damages equally 
        among tortfeasors) should give way to partial (or comparative) 
        indemnity, ``under which liability among multiple tortfeasors may be 
        apportioned on a 
        comparative negligence basis.''\footnote{Casebook p. 470.}
        \item Comparative indemnity is allowed by statute.\footnote{Casebook 
        p. 471.}
        \item Defendants can cross-claim against any person, ``whether already a 
        party or not, from whom the named defendant seeks to obtain total or 
        partial indemnity.''\footnote{Casebook p. 471.}
    \end{enumerate}
    \item AMA's cross-complaint was allowed.
\end{enumerate}

\subsubsection{Proposition 51: Fair Responsibility Act of 1986}

\begin{enumerate}
    \item Prop. 51 \textbf{eliminated joint and several liability for 
    non-economic damages}---i.e., defendants are only liable for non-economic 
    damages in proportion to their share of the fault.
    \item Defendants are still jointly and severally liable for economic 
    damages.
    \item Example: if total non-economic damages are \$200,000, and defendant 
    B is 30\% liable, the plaintiff can recover \$60,000 in non-economic 
    damages from B. If economic damages are \$100,000, the plaintiff can still 
    recover the full \$100,000 from B.
\end{enumerate}
